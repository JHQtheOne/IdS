\documentclass[10pt,a4paper,landscape]{article}
\usepackage{pdflscape}
\usepackage[utf8]{inputenc}
\usepackage[ngerman]{babel}
\usepackage[left=2.5cm,right=2.5cm,top=4cm,bottom=3cm]{geometry} % Seitenränder 
\usepackage{amsmath ,amsfonts, amssymb}
\usepackage{graphicx}
\usepackage{mathabx} % contains \Dashv
\usepackage{enumitem}
\usepackage{titling} % provides \thetitle, \theauthor and \thedate 
\usepackage{fancyhdr, fancybox}
\usepackage{tikz, tikz-qtree}
\usepackage{listings}


\newcommand{\exnum}{9} % define variable exnum
\title{Informatik der Syteme \\ Exercise \exnum}
\author{Jan Hoffmann, Matr. 3177642 \\ Mike Hengge,  Matr. 3940400}

\pagestyle{fancy}
% Header
\fancyhead[L]{\sffamily\textbf{\thetitle}}
\fancyhead[C]{}
\fancyhead[R]{\sffamily{\theauthor}}
% Footer
\fancyfoot[L]{}
\fancyfoot[C]{}
\fancyfoot[R]{\thepage}
\begin{document}
\section{}
	\begin{enumerate}
	\begin{landscape}
		\item 
			\setlength{\tabcolsep}{0.8mm}
			\begin{tabular}{l || c c c c c c c c c c c c c c c c c c c c c c c c c} 
			
  Instruktionen & 1 & 2 & 3 & 4 & 5 & 6 & 7 & 8 & 9 & 10 & 11 & 12 & 13 & 14 & 15 & 16 & 17 & 18 & 19 & 20 & 21 & 22 & 23 & 24 & 25\\ 
	\hline \hline
  lw r1, 0x1000(r0) 	& IF & ID & EX & MA & WB \\
	lw r2, 0x1004(r0) 	&    & IF & ID & EX & MA & WB \\
	add r3, r1, r2    	&    &    & IF & ID &    &    & EX & MA & WB\\
	sub r4, r1, r2    	&    &    &    & IF &    &    & ID & EX & MA & WB\\
	sw 0x100c(r0), r4		&    &    &    &    & IF &    &    &    &    &    & ID & EX & MA & WB\\
	sw 0x1008(r0), r3   &    &    &    &    &    & IF &    &    &    &    &    & ID & EX & MA & WB \\
	addi r5, r0, 0x10   &    &    &    &    &    &    & IF &    &    &    &    &    & ID & EX & MA & WB   \\
	sub r6, r0, r5			&    &    &    &    &    &    &    & IF &    &    &    &    &    &    &    &    & ID & EX & MA & WB \\
	sw 0x2000(r1), r5   &    &    &    &    &    &    &    &    & IF &    &    &    &    &    &    &    &    & ID & EX & MA & WB   \\
	sw 0x2000(r2), r6   &  &   &   &   &   &   &   &    &    & IF &    &    &    &    &    &    &    &    &    &    &    & ID & EX & MA & WB  \\
			\end{tabular}
				\item Um 10 Takte. (25 tatsächlich - 15 mindestens)
		\item Mit Pipelining: 25 * 200ps = 5 ns \\
					Ohne Piplining: 10 * 800ps = 8 ns \\
				  Hier wurde also eine Beschleunigung von 37,5\% erzielt.
			\end{enumerate}
			\end{landscape}
\section{}
	\begin{enumerate}
	\begin{landscape}	
		\item 
			
			\begin{tabular}{l || c c c c c c c c c c c c c c c c c c} 
  Instruktionen & 1 & 2 & 3 & 4 & 5 & 6 & 7 & 8 & 9 & 10 & 11 & 12 & 13 & 14 & 15 & 16 & 17 & 18\\ 
	\hline \hline
  lw r1, 0x1000(r0) 	& IF & ID & EX & MA & WB \\
	lw r2, 0x1004(r0) 	&    & IF & ID & EX & MA & WB \\
	addi r5, r0, 0x10   &    &    & IF & ID & EX & MA & WB\\
	add r3, r1, r2    	&    &    &    & IF & ID &    &    & EX & MA & WB\\
	sub r6, r0, r5			&    &    &    &    & IF & ID &    &    & EX & MA & WB\\
	sw 0x2000(r1), r5   &    &    &    &    &    & IF &    &    & ID & EX & MA & WB\\
	sub r4, r1, r2    	&    &    &    &    &    &    & IF &    &    &    & ID & EX & MA & WB\\
	sw 0x1008(r0), r3		&    &    &    &    &    &    &    & IF &    &    &    & ID & EX & MA & WB \\
	sw 0x2000(r2), r6   &    &    &    &    &    &    &    &    & IF &    &    &    & ID & EX & MA & WB\\
	sw 0x100c(r0), r4		&    &    &    &    &    &    &    &    &    & IF &    &    &    &    & ID & EX & MA & WB\\   
	
			\end{tabular}
		 
		\item  Die Pipeline muss um 3 Takte verzögert werden (18 statt 15).
		\item  Mit Pipeline: 18 * 200ps = 3,6 ns. \\
					 Ohne Pipeline: 10 * 800ps = 8,0 ns. \\
					 Hier wurde also eine Beschleunigung von 55\% erreicht.
				
			\end{enumerate}
			
\end{document}
