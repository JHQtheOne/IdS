\documentclass[10pt,a4paper]{article}
\usepackage[latin1]{inputenc}
\usepackage[ngerman]{babel}
\usepackage{amsmath}
\usepackage{amsfonts}
\usepackage{amssymb}
\usepackage{graphicx}
\usepackage{mathabx} % contains \Dashv
\usepackage{fancyhdr}

\title{Informatik der Syteme - Exercise 2}
\author{Jan Hoffmann}
\pagestyle{fancy}
% Header
\fancyhead[L]{\textbf{Exercise 2}}
\fancyhead[C]{\textbf{Informatik der Syteme}}
\fancyhead[R]{Jan Hoffmann\\ Matr. 3177642}
% Footer
\fancyfoot[L]{}
\fancyfoot[C]{}
\fancyfoot[R]{\thepage}

\begin{document}
\section*{Problem 2.1:}
\begin{itemize}
  \item[1.]
		$X = 44/7 \approx 2 \cdot \pi \approx 6,2857142857142857142857142857143 $\\
		$X$ ist positiv $\Rightarrow s = 0$ \\
		$6,2857142857142857142857142857143 $\\
		$= 3,1428571428571428571428571428571 \cdot 2^1 $\\
		$= 1,5714285714285714285714285714286 \cdot 2^2 $\\
		$\Rightarrow f = 0,5714285714285714285714285714286$\\
		$e = 2 + B = 2 + 15 = 17 $\\
		$X = (-1)^{0} \cdot (1 + 0,5714285714285714285714285714286) \cdot 2^{17-15} $	
		
	\item[2.]		
		$s_2 = 0$\\
		$e_2 = 10001$\\
		$0,5714285714285714285714285714286 \cdot 2 = 1 + 0,1428571428571428571428571428571$\\
		$1,1428571428571428571428571428571 \cdot 2 = 0 + 0,2857142857142857142857142857143$\\
		$0,2857142857142857142857142857143 \cdot 2 = 0 + 0,5714285714285714285714285714286$\\
		$0,5714285714285714285714285714286 \rightarrow$ periodisch\\
		$f_2 = 1001001001$\\
		
	\item[3.]
		$X = \underbrace{0}_{s}\underbrace{10001}_{e}\underbrace{1001001001}_{f}$
	
	\item[4.]
		Mit dieser Anordnung lassen sich IEEE 754 Zahlen, genau wie signed Integer Werte, lexikalisch ordnen und damit vergleichen.
		Mit anderen Anordnungen m�sste man die Zahl aufsplitten und die Teile einzeln vergleichen.
		
	\item[5.]
	  $X_2    = 0100 \, 0110 \, 0100 \, 1001$\\
		$X_{16} =   4        6     4       9$\\
\end{itemize}
		
\section*{Problem 2.2:}
	$X = ABCD_{16} = 1010 \, 1011 \, 1100 \, 1101_2$\\
	$s = 1 \Rightarrow X$ negativ\\
	$e_2 = 01010 \Rightarrow e_{10} = 10 \Rightarrow e = 10 - 15 = -5$\\
	$f_2 = 0,1111001101 \Rightarrow f_{10} = 0,9501953125 $\\
	$(= 1 \cdot 2^{-1} + 1 \cdot 2^{-2} + 1 \cdot 2^{-3} + 1 \cdot 2^{-4} + 0 \cdot 2^{-5} + 0 \cdot 2^{-6} + 1 \cdot 2^{-7} + 1 \cdot 2^{-8} + 0 \cdot 2^{-9} + 1 \cdot 2^{-10})$\\
	$X = (-1)^{1} \cdot (1 + 0,9501953125) \cdot 2^{-5} $ \\
	$X = -0,060943603515625 $\\
	

	
\end{document}