\documentclass[10pt,a4paper]{article}
\usepackage[utf8]{inputenc}
\usepackage[ngerman]{babel}
\usepackage[left=2.5cm,right=2.5cm,top=4cm,bottom=3cm]{geometry} % Seitenränder 
\usepackage{amsmath}
\usepackage{amsfonts}
\usepackage{amssymb}
\usepackage{graphicx}
\usepackage{mathabx} % contains \Dashv
\usepackage{enumitem}
\usepackage{fancyhdr}

\title{Informatik der Syteme - Exercise 5}
\author{Jan Hoffmann}
\pagestyle{fancy}
% Header
\fancyhead[L]{\textbf{Informatik der Systeme}\\ \textbf{Exercise 5}}
\fancyhead[C]{}
\fancyhead[R]{Jan Hoffmann, Matr. 3177642\\ Mike Hengge, Matr. 3940400}
% Footer
\fancyfoot[L]{}
\fancyfoot[C]{}
\fancyfoot[R]{\thepage}

\begin{document}
\section*{Problem 5.1:}
\begin{enumerate}
  \item $\setlength{\arraycolsep}{0pt}
				\begin{array}[t]{*{9}{l}}
					1&0&0&0&0&0&0&0&0&1\\
					1&0&1&1&1& & & & & \\\cline{1-5}
					 & &1&1&1&0&0& & & \\
					 & &1&0&1&1&1& & & \\\cline{3-7}
					 & & &1&0&1&1&0& & \\
					 & & &1&0&1&1&1& & \\\cline{4-8}
					 & & & & & & &1& &
				\end{array}:10111=1011 $ \\\\
				q = 2^4 -1 = 7 \\
				\Rightarrow 0 < q < 2^p-1 \Rightarrow 0 < q < 15 \\
				$ Die Periode von G(u) ist 7.
		\item $n = m + k \Rightarrow m = n - k \\
					m = 15 -4 = 11 \Rightarrow $ Es sind 11 Nutzbits.
		\item Prüfstellen für 001: $001 * 10111 = 10111$
		\item Primitive Teiler von G(u): \\
					11 und 1101.
		\item 
		
				
\end{enumerate}

\section*{Problem 5.2:}
\begin{enumerate}
	\item 
	\item 
\end{enumerate}

\section*{Problem 5.3:}
\begin{enumerate}	
	\item CPI_e_f_f $= 1 * 0,5 + 5 * 0,2 + 3 * 0,1 + 2 * 0,2 = 2,2 
	\item 
	\item CPI_e_f_f $= 1 * 0,5 + 5 * 0,2 + 3 * 0,1 + 2* 0,1 = 2,0 \\
				\Rightarrow $\frac{2,2}{2,0} = 1,1$ \Rightarrow  $ 10\% schneller
	\item CPI_e_f_f $= 1 * 0,25 + 5 * 0,2 + 3 * 0,1 + 2 * 0,2 = 1,95 \\
				\Rightarrow $\frac{2,2}{1,95} \approx 1,13 $ \Rightarrow $ 13\% schneller
\end{enumerate}

\end{document}