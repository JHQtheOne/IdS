\documentclass[10pt,a4paper]{article}
\usepackage[utf8]{inputenc}
\usepackage[ngerman]{babel}
\usepackage[left=2.5cm,right=2.5cm,top=4cm,bottom=3cm]{geometry} % Seitenränder 
\usepackage{amsmath}
\usepackage{amsfonts}
\usepackage{amssymb}
\usepackage{graphicx}
\usepackage{mathabx} % contains \Dashv
\usepackage{enumitem}
\usepackage{fancyhdr}

\title{Informatik der Syteme - Exercise 6}
\author{Jan Hoffmann, Mike Hengge}
\pagestyle{fancy}
% Header
\fancyhead[L]{\textbf{Informatik der Systeme}\\ \textbf{Exercise 6}}
\fancyhead[C]{}
\fancyhead[R]{Jan Hoffmann, Matr. 3177642\\ Mike Hengge, Matr. 3940400}
% Footer
\fancyfoot[L]{}
\fancyfoot[C]{}
\fancyfoot[R]{\thepage}

\begin{document}
\section*{Problem 6.1:}
	\begin{verbatim}
		1       100011 00000 00001 00010 00000000000 # lw r1, 0x1000 (r0) // r1 = input				
		2       000000 00001 00000 00010 00000100000 # add r2, r1, r0     // r2 = r1					
		3       001000 00000 00011 0000000000000001  # addi r3, r0, 1     // r3 = 1						
		4 loop: 000100 00010 00000 0000000000001100  # beqz r2, end       // if r2=0 goto end	
		5       000000 00010 00011 00011 00000011001 # multu r3, r2, r3   //   r3 = r2 * r3		
		6       001010 00010 00010 0000000000000001  # subi r2, r2, 1     //   r2 = r2 - 1		
		7       000010 11111111111111111111110000    # j loop             // goto loop				
		8 end:  101011 00000 00011 0001000000000100  # sw r3, 0x1004 (r0) // output = r3			
	\end{verbatim}			
	Dieser Code implementiert die Fakultätsfunktion. \\
	Der Counter r2 wird von dem Wert, der in Register 0x1000 steht, dekrementiert bis er 0 ist. \\
	Bei jedem Schleifendurchlauf wird r2 auf das Endergebnis r3 multipliziert.\\
	$(n \cdot n-1 \cdot ... \cdot 1) = n!$

\section*{Problem 6.2:}
	\begin{verbatim}
		 1 lw r1 , 0 x1000 (r0)     # 10001100000000010001000000000000 ; 0x8c011000 
		 2 lw r2 , 0 x1004 (r0)     # 10001100000000100001000000000100 ; 0x8c021004 
		 3 loop : beqz r1 , end     # 00010000001000000000000000010010 ; 0x10200012 
		 4 slt r3, r1, r2           # 00000000001000100001100000101010 ; 0x0022182a 
		 5 bnez r3 , branch         # 00010000011000000000000000001110 ; 0x1060000d 
		 6 sub r3 , r1 , r2         # 00000000001000100001100000100010 ; 0x00221822 
		 7 add r1 , r2 , r0         # 00000000010000000000100000100000 ; 0x00400820 
		 8 add r2 , r3 , r0         # 00000000011000000001000000100000 ; 0x00601020 
		 9 j loop                   # 00001011111111111111111111110000 ; 0x0bfffff0 
		10 branch: sub r3, r2, r1   # 00000000010000010001100000100010 ; 0x00411802 
		11 add r2, r1, r0           # 00000000001000000001000000100000 ; 0x00201020 
		12 add r1, r3, r0           # 00000000011000000000100000100000 ; 0x00600820 
		13 j loop                   # 00001011111111111111111111110000 ; 0x0bfffff0 
		14 end: sw 0x1008 (r0), r2  # 10101100000000100001000000001000 ; 0xac021008 
	\end{verbatim}

\end{document}