\documentclass[10pt,a4paper]{article}
\usepackage[utf8]{inputenc}
\usepackage[ngerman]{babel}
\usepackage[left=2.5cm,right=2.5cm,top=4cm,bottom=3cm]{geometry} % Seitenränder 
\usepackage{amsmath ,amsfonts, amssymb}
\usepackage{graphicx}
\usepackage{mathabx} % contains \Dashv
\usepackage{enumitem}
\usepackage{titling} % provides \thetitle, \theauthor and \thedate 
\usepackage{fancyhdr, fancybox}
\usepackage{tikz, tikz-qtree}
\usepackage{listings}

\renewcommand\thesection{\sffamily{Problem \exnum.\arabic{section}:}\\}
\newcommand{\exnum}{10} % define variable exnum
\title{Informatik der Syteme \\ Exercise \exnum}
\author{Jan Hoffmann, Matr. 3177642 \\ Mike Hengge,  Matr. 3940400}

\pagestyle{fancy}
% Header
\fancyhead[L]{\sffamily\textbf{\thetitle}}
\fancyhead[C]{}
\fancyhead[R]{\sffamily{\theauthor}}
% Footer
\fancyfoot[L]{}
\fancyfoot[C]{}
\fancyfoot[R]{\thepage}

\begin{document}
\section{}\label{sec:10.1}
	\begin{enumerate}
		\item
	\end{enumerate}

\section{Cache}\label{sec:10.2}
	\begin{enumerate}
		\item
		\item
		\item
		\item \text{}\\
			\begin{tabular}{cc|lllll}
				\# & Anfrage         & Zelle 0 & Zelle 1 & Zelle 2 & Zelle 3 & Kommentar              \\ \hline
				   & urspr. Belegung & 8:88    & 5:50(*) & 2:22    & 7:70(*) &                        \\
				 1 & MEM[1] = 10     & 8:88    & 1:10(*) & 2:22    & 7:70(*) & write miss, write back \\
				 2 & read MEM[4]     & 4:44    & 1:10(*) & 2:22    & 7:70(*) & read miss              \\
				 3 & read MEM[7]     & 4:44    & 1:10(*) & 2:22    & 7:70(*) & read hit                       \\ 
				 4 & read MEM[3]     & 4:44    & 1:10(*) & 2:22    & 3:33    & read miss, write back  \\ 
				 5 & MEM[2] = 20     & 4:44    & 1:10(*) & 2:20(*) & 3:33    & write miss             \\
				 6 & MEM[0] = 0      & 0:00(*) & 1:10(*) & 2:20(*) & 3:33    & write miss             \\ 
				 7 & MEM[3] = 30     & 0:00(*) & 1:10(*) & 2:20(*) & 3:30(*) & write miss             \\ 
				 8 & read MEM[0]     & 0:00(*) & 1:10(*) & 2:20(*) & 3:30(*) & read hit               \\ 
				 9 & read MEM[6]     & 0:00(*) & 1:10(*) & 6:66    & 3:30(*) & read miss, write back  \\ 
				10 & read MEM[5]     & 0:00(*) & 5:55    & 6:66    & 3:30(*) & read miss, write back  \\ 
			\end{tabular}
	\end{enumerate}
			
\end{document}
