\documentclass[10pt,a4paper]{article}
\usepackage[utf8]{inputenc}
\usepackage[ngerman]{babel}
\usepackage[left=2.5cm,right=2.5cm,top=4cm,bottom=3cm]{geometry} % Seitenränder 
\usepackage{amsmath}
\usepackage{amsfonts}
\usepackage{amssymb}
\usepackage{graphicx}
\usepackage{mathabx} % contains \Dashv
\usepackage{enumitem}
\usepackage{fancyhdr}
\usepackage{fancybox}
\usepackage{tikz}
\usepackage{tikz-qtree}
\usepackage{listings}
\usepackage{mips}

\lstset{ %
  language=[mips]Assembler,       % the language of the code
  basicstyle=\footnotesize,       % the size of the fonts that are used for the code
  numbers=left,                   % where to put the line-numbers
  numberstyle=\tiny\color{gray},  % the style that is used for the line-numbers
  stepnumber=1,                   % the step between two line-numbers. If it's 1, each line 
                                  % will be numbered
  numbersep=5pt,                  % how far the line-numbers are from the code
  backgroundcolor=\color{white},  % choose the background color. You must add \usepackage{color}
  showspaces=false,               % show spaces adding particular underscores
  showstringspaces=false,         % underline spaces within strings
  showtabs=false,                 % show tabs within strings adding particular underscores
  frame=single,                   % adds a frame around the code
  rulecolor=\color{black},        % if not set, the frame-color may be changed on line-breaks within not-black text (e.g. commens (green here))
  tabsize=4,                      % sets default tabsize to 2 spaces
  captionpos=b,                   % sets the caption-position to bottom
  breaklines=true,                % sets automatic line breaking
  breakatwhitespace=false,        % sets if automatic breaks should only happen at whitespace
  title=\lstname,                 % show the filename of files included with \lstinputlisting;
                                  % also try caption instead of title
  keywordstyle=\color{blue},          % keyword style
  commentstyle=\color{dkgreen},       % comment style
  stringstyle=\color{mauve},         % string literal style
  escapeinside={\%*}{*)},            % if you want to add a comment within your code
  morekeywords={*,...}               % if you want to add more keywords to the set
}

\title{Informatik der Syteme - Exercise 7}
\author{Jan Hoffmann, Mike Hengge}
\pagestyle{fancy}
% Header
\fancyhead[L]{\textbf{Informatik der Systeme}\\ \textbf{Exercise 7}}
\fancyhead[C]{}
\fancyhead[R]{Jan Hoffmann, Matr. 3177642\\ Mike Hengge, Matr. 3940400}
% Footer
\fancyfoot[L]{}
\fancyfoot[C]{}
\fancyfoot[R]{\thepage}

\begin{document}
\section*{Problem 7.1:}
	\item 
		\lstinputlisting {fib.asm}
\section*{Problem 7.2:}
	\item Für die iterative Darstellung werden 10 Register benötigt. \\
				Für die rekursive Darstellung werden 4 Register benötigt.
	
\section*{Problem 7.3:}
	\item Es werden 4 Funktionsaufrufe benötigt um auf das 4. Fibonacci-Element zu kommen. \\\\\\
		\begin{tikzpicture}[grow=down]
			\Tree [.fib(4)	[.fib(3) [.fib(2) [.fib(1) 1 ][.fib(0) 0 ]][.fib(1) 1 ] ] 
											[.fib(2) [.fib(1) 1 ][.fib(0) 0 ] ] ]
		\end{tikzpicture}	
	
\section*{Problem 7.4:}
\item
	In der tiefsten Rekursionsstufe ist der Stack am größten (tiefste Ebene im Baum, s.o.).
	Für jeden Aufruf von recurse werden drei Register auf den Stack gepusht (\$sp wird um 12 veringert).
	Dann wird die Return Address \$ra in \$sp + 0 und \$a0 in \$sp + 4 gespeichert. Das passiert bis \$a0 = 1 ist. 
	Dann wird direkt ein Wert (1) zurückgegeben und mit der Rekursion für \$a0 - 2 begonnen. Dafür ist jeweils \$sp + 8 reserviert.
	\\
	
	\begin{tabular}{|l|l|}
		\hline
		address			&	value		\\ \hline	
		\$sp_0 - 36	&	\$ra_1	\\
		\$sp_0 - 32	&	2				\\
		\$sp_0 - 28	&	0				\\ \hline
		\$sp_0 - 24	&	\$ra_1	\\
		\$sp_0 - 20	&	3				\\
		\$sp_0 - 16	&	0				\\ \hline
		\$sp_0 - 12	&	\$ra_0	\\
		\$sp_0 - 8	&	4				\\
		\$sp_0 - 4	&	0				\\ \hline
		\$sp_0			& start value for stack pointer\\	\hline
	\end{tabular}
	
	\item
	Auf dem Rückweg werden in jeder Rekursion die Summe der beiden Kinder eines Knotens gebildet und in \$sp + 8 gespeichert.
	Danach werden wieder drei Register gepopt in dem \$sp um 12 erhöht wird.
	

\end{document}
