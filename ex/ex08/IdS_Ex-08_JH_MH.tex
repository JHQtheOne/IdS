\documentclass[10pt,a4paper]{article}
\usepackage[utf8]{inputenc}
\usepackage[ngerman]{babel}
\usepackage[left=2.5cm,right=2.5cm,top=4cm,bottom=3cm]{geometry} % Seitenränder 
\usepackage{amsmath ,amsfonts, amssymb}
\usepackage{graphicx}
\usepackage{mathabx} % contains \Dashv
\usepackage{enumitem}
\usepackage{titling} % provides \thetitle, \theauthor and \thedate 
\usepackage{fancyhdr, fancybox}
\usepackage{tikz, tikz-qtree}
\usepackage{listings}


\newcommand{\exnum}{8} % define variable exnum
\title{Informatik der Syteme \\ Exercise \exnum}
\author{Jan Hoffmann, Matr. 3177642 \\ Mike Hengge,  Matr. 3940400}

\pagestyle{fancy}
% Header
\fancyhead[L]{\sffamily\textbf{\thetitle}}
\fancyhead[C]{}
\fancyhead[R]{\sffamily{\theauthor}}
% Footer
\fancyfoot[L]{}
\fancyfoot[C]{}
\fancyfoot[R]{\thepage}
	
\renewcommand\thesection{\sffamily{Problem \exnum.\arabic{section}:}\\}

\lstdefinestyle{Assembler}{ %
  language=[mips]Assembler,       % the language of the code
  basicstyle=\scriptsize\ttfamily,       % the size of the fonts that are used for the code
  numbers=left,                   % where to put the line-numbers
  numberstyle=\tiny\color{gray},  % the style that is used for the line-numbers
  stepnumber=1,                   % the step between two line-numbers. If it's 1, each line 
                                  % will be numbered
  numbersep=5pt,                  % how far the line-numbers are from the code
  backgroundcolor=\color{white},  % choose the background color. You must add \usepackage{color}
  showspaces=false,               % show spaces adding particular underscores
  showstringspaces=false,         % underline spaces within strings
  showtabs=false,                 % show tabs within strings adding particular underscores
  frame=single,                   % adds a frame around the code
  rulecolor=\color{black},        % if not set, the frame-color may be changed on line-breaks within not-black text (e.g. commens (green here))
  tabsize=4,                      % sets default tabsize to 2 spaces
  captionpos=b,                   % sets the caption-position to bottom
  breaklines=true,                % sets automatic line breaking
  breakatwhitespace=false,        % sets if automatic breaks should only happen at whitespace
  title=\lstname,                 % show the filename of files included with \lstinputlisting;
                                  % also try caption instead of title
  keywordstyle=\color{blue},          % keyword style
  commentstyle=\color{dkgreen},       % comment style
  stringstyle=\color{mauve},         % string literal style
  escapeinside={\%*}{*)},            % if you want to add a comment within your code
  morekeywords={*,...}               % if you want to add more keywords to the set
}

\lstdefinestyle{DOS}{backgroundcolor=\color{black}, basicstyle=\scriptsize\color{white}\ttfamily, xleftmargin = \parindent}
\lstdefinestyle{C}{basicstyle=\scriptsize\ttfamily, numbers=left, numberstyle=\tiny, numbersep=8pt, frame = shadowbox, language = C, framexleftmargin=15pt, xleftmargin = \parindent}

\begin{document}
\section{}
	\begin{enumerate}
		\item 		
			\begin{lstlisting}[style=DOS] 
> gcc -save-temps main.c
			\end{lstlisting}
			erzeugt folgenden preprocessor code:
			\lstinputlisting[style=C, label=main.i]{main.i}
		
			Im Vergleich zum Original Code wurde die Konstante UPPER durch ihren Wert 15 ersetzt.
		
			Der Präprozessor vereinfacht den Code (optimiert für den Compiler), um möglichst effizient in Assembler Code übersetzen zu können.
			
		\item 
			$i = lower:$ \\
			\begin{lstlisting}[style=Assembler]
movl	_lower, %eax
movl	%eax, 12(%esp)
			\end{lstlisting}
			Speichert \_lower auf den Stack: 12(\%esp) ist das Register das i repräsentiert.
			
			$sum += i$ \\
			\begin{lstlisting}[style=Assembler]
movl	_sum, %edx
movl	12(%esp), %eax
addl	%edx, %eax
movl	%eax, _sum
			\end{lstlisting}
			load sum into \%edx\\
			load i from Stack into \%eax\\
			add \%edx to \%eax \\
			store \%eax into sum \\
			
			$i < UPPER:$\\
			\begin{lstlisting}[style=Assembler]
cmpl	$14, 12(%esp)
jle	L3			
			\end{lstlisting}
			Jump to L3 if i $\leq$ 14 ($\widehat{=}$ if $i < 15$)
		
		\item
			Da das Programm nur mit Konstanten rechnet, steht das Ergebnis schon fest. 
			Der Kompiler rechnet also die 3 Schleifendurchläufe (12+13+14) = 39 schon mal aus und speichert nur das Ergebnis.
			Der optimierte Code berechnet solche Zwischenergebnisse zur Kompiliertzeit und erreicht damit eine höhere Effizient zur Laufzeit.
		
		\item
			
	\end{enumerate}
\section{}
	

\end{document}
