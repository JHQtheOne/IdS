\documentclass{scrartcl}
\usepackage[utf8]{inputenc}
\usepackage{scrpage2}
\usepackage{rotating}
\usepackage{listings}
\usepackage{amsmath}
\usepackage{mathtools}
\usepackage{hyperref}
\pagestyle{scrheadings}
%alles was über diesem Kommentar steht ist so über die jahre gewachsen, das sind (bis jetzt) alle Packages die ich gebraucht habe. Ich ändere da eigentlich nie etwas daran 


\renewcommand{\thesubsection}{\alph{subsection}} %hier habe ich die subsection beschriftung geändert. kommentiert den befehl einfach aus und guckt was euch besser gefällt. 

\ohead[]{Ich stehe im Blattkopf}
\cfoot[\pagemark]{\pagemark}
\author{Ich stehe im TITEL}
\title{TITEL DES BLATTES }
\begin{document}
\maketitle %hier wird der titel geschrieben 
\section{Aufgabe}
\subsection{Teilaufgabe}
\[ 3x = 2x +1\]
\[ x = 1 \]
\subsection{Teilaufgabe}
\[ \in \frac{2x}{1}  \sum_{n=0}^N  \] 
HAllo ich bin text $ich\ bin text im\ Mathe Modus.$ \\
Jeder Absatz muss extra geschrieben werden, umbrüche sind aber ganz gut \\
Tabelle ohne strichen \\
\begin{tabular}{ l c r }
  1 & 2 & 3 \\
  4 & 5 & 6 \\
  7 & 8 & 9 \\
\end{tabular}
\\
Tabelle mit strichen
\begin{tabular}{ l| c| r }
  1 & 2 & 3 \\ \hline
  4 & 5 & 6 \\ \hline
  7 & 8 & 9 \\
\end{tabular}
\\
Tabelle mit Mathe symbolen 
\begin{tabular}{ l | c | r }
  1 & 2 & $\in$ \\ \hline
  4 & $\leq$ & 6 \\ \hline
  7 & 8 & 9 \\ 
\end{tabular}
\\
Bäume und Bilder mache ich immer mit Inkscape und binde diese dann hier ein. Wie das geht zeige ich euch wenn ihr das mal Braucht
\newpage
Ich bin auf der nächsten seite
\end{document}
